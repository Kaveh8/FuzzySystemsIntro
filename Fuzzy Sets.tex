\documentclass[12pt,a4paper]{article}

\usepackage{amssymb}
\usepackage[scale=0.75]{geometry} 
\usepackage[pagebackref=true,colorlinks,linkcolor=blue,citecolor=magenta]{hyperref}
\usepackage{tocbibind}
\usepackage{xepersian}

\settextfont[Scale=1]{XB Niloofar}
\setlatintextfont{Linux Libertine}

\setlength{\parskip}{1em}
\renewcommand{\baselinestretch}{1.5}
\title{‌سیستم‌های فازی و هوش مصنوعی}
\author{کاوه شاه‌حسینی}

\begin{document}
%*******************Cover Page****************************
\thispagestyle{empty}
\vspace*{-28mm}
\centerline{\includegraphics[height=4cm]{pnu-logo.jpg}}
\begin{center}
	\vspace{-2mm}
	{\large
	دانشگاه پیام نور تهران مرکز شمیرانات
	\\[.1cm]
	دانشکده مهندسی کامپیوتر
	\\[.1cm]
		مهندسی نرم‌افزار
	\\[1.7cm]
		عنوان سمینار
	\\[.9cm]
	}
	{\Huge 
		سیستم‌های فازی و هوش مصنوعی
	}
	\\[1.8cm]
	استاد 
	\\[.2cm]
	\textbf{\large {دکتر مهدی خلیلی}}
	\\[1.5cm]
	دانشجو
	\\[.1cm]
	\textbf{\large { کاوه شاه‌حسینی}}
	\\[1.5cm]
	شماره دانشجویی
	\\[.1cm]
	\textbf{\large {۹۲۳۸۴۳۰۱۰‌}}
	\\[1.5cm]
	{\large
		ترم ۳۹۵۱
	}
\end{center}
%********************Table of Content**********************
\newpage
\baselineskip=1cm
\tableofcontents
\newpage
\listoftables
\baselineskip=.75cm
\newpage 
%*******************Fuzzy Sets****************************
\section{مجموعه‌های فازی}
 \subsection{مجموعه‌های کلاسیک}
پیش از آنکه به توضیح مجموعه‌های فازی بپردازیم، نیاز است تا مروری بر مجموعه‌های کلاسیک داشته باشیم. همه‌ی ما کم و بیش با مفهموم مجموعه آشنا هستیم. به عنوان مثال مجموعه‌ای از دانشجویان یک رشته، مجموعه‌ای از اعداد بزرگ‌تر از صفر و... نمونه‌ای از مجموعه‌ها هستند. مجموعه‌ها گروهی از اشیاء متمایز هستند. به اشیاء درون هر مجموعه، اعضاء و یا عناصر آن مجموعه گفته می‌شود. \\
در ریاضیات مجموعه را با حروف بزرگ 
$\lr{A,B,C,...}$
 و اعضای مجموعه را با حروف کوچک 
$\lr{a,b,c,...} $
  نمایش می‌دهند. همچنین عضویت یک شیء در مجموعه با نماد 
$\in$
 و عدم عضویت با نماد
 $\notin$
 نمایش داده می‌شود. 
 \cite{Bojadziev2007}
 \\
 اگر مجموعه $ U $  را به عنوان مجموعه‌ی جهانی یا مرجع در نظر بگیریم، که شامل همه‌ی اشیاء موجود در یک کاربرد مشخص هستند، آنگاه نمایش مجموعه $ A $  در این مجموعه‌ی مرجع می‌تواند با لیست کردن تمامی عناصر آن صورت گیرد و یا با استفاده از روش قانون عضویت که در 
 \ref{f_1}
 آمده است، تعریف شود.
 \cite{Wang1997}
 \begin{equation}\label{f_1}
 A= \{\  x \in U\ |\   x\ {\rm meets\ some\ conditions\ } \}
 \end{equation} 
 در مجموعه‌های کلاسیک عضویت یک شیء تنها دو حالت دارد بدین صورت که شیء $x$ یا متعلق به مجموعه $A$ هست و یا نیست. این مجموعه‌ها را به دلیل آنکه فضای آن با دقت ۱۰۰٪ قابل تشخیص است و عناصر آن دارای ارزش عضویت صفر و یا یک هستند، مجموعه‌های
  \textbf{قطعی}
  نیز می‌گویند. 
\begin{equation}\label{f_2}
\mu_{A}(x)=\left \{ 
	{1 \atop 0}
	\hskip 1cm
   {if \hskip 1cm x \in A, \atop
	if \hskip 1cm x \notin A.} \right.
\end{equation} 
در اینجا
$\mu_{A}(x)$
 تابع عضویت مجموعه‌ی قطعی $A$ می‌باشد. تابع عضویت، تابعی است که ارزش عضویت یک شیء را در یک مجموعه، مشخص می‌کند. طبق فرمول 
 \ref{f_2}
 تابع عضویت مجموعه‌های قطعی، تنها دو مقدار یک و یا صفر را برای مشخص کردن عضویت و یا عدم عضویت یک شیء در  مجموعه، بازمی‌گرداند. \\
 مجموعه‌ای که دارای هیچ عضوی نیست را مجموعه‌ی تهی می‌نامند و با علامت $\varnothing$ نمایش داده می‌شود. مجموعه‌ی تهی زیرمجموعه‌ی هر مجموعه‌ای است.
 \cite{Lee2005}
 \\
فرض کنید دو مجموعه‌ی $ A $ و $ B  $ در مجموعه‌ی مرجع $ U $ وجود دارد.  در جدول   
\ref{table:t_1}
\textbf{روابط بین مجموعه‌ها} 
 آورده شده است:
\begin{table}[!htbp]
\begin{center}
	{\footnotesize
 \begin{tabular}{r c c r} \hline
عنوان & نماد &  مثال & توضیح  
\\\hline 
%**********
زیر مجموعه  &
$\subseteq$ &
$A \subseteq B$ &
همه‌ی اعضای $A$ در $B$ نیز هست و یا دو مجموعه با یکدیگر مساوی هستند.
\\
%*********
زیر مجموعه سره &
$\subset$ &
 $A \subset B$ &
همه‌ی اعضای $A$ در $B$ نیز هست. ولی حداقل یک عضو در $B$ هست که در $A$ نیست.
\\
%**********
مساوی &
$=$ &
$A = B$ &
همه‌ی اعضای $A$ در $B$ نیز هست و همه‌ی اعضای $B$ نیز در $A$ هست.
\\\hline
 %**********
 \end{tabular}
 \caption{روابط بین مجموعه‌ها}
 \label{table:t_1}
}
\end{center}
\end{table}
همچنین در جدول
\ref{table:t_2}
\textbf{عملیات بر روی مجموعه‌ها} 
 آورده شده‌ است:
\begin{table}[h]
	{\footnotesize
	\begin{center}
		\begin{tabular}{r c l} \hline
			عنوان & نماد & تعریف  
			\\\hline 
			%**********
			اجتماع   &
			$\cup$ &
			$A \cup B = \{x\ |\ x \in A\ {\rm or}\ x \in B \}$ 
			\\
			%*********
			اشتراک   &
			$\cap$ &
			$A \cap B = \{x\ |\ x \in A\ {\rm and}\ x \in B \}$ 
			\\
			%**********
			تفاضل &
			$-$ &
			$A - B = \{x\ |\ x \in A\ {\rm and}\ x \notin B \}$ 
		   \\
			%**********
			مکمل &
			$\overline{A}$ &
			$\overline{A} = \{x\ |\ x \notin A{\rm ,}\ x \in U \}$ 
			\\\hline 
			%**********
		\end{tabular}
		\caption{عملیات بر روی مجموعه‌ها}
		\label{table:t_2}
\end{center}
	}
\end{table} 
\\
%***************************************************************************
 \subsection{تعریف مجموعه‌های فازی}
در سال ۱۹۶۵ پروفسور لطفی‌زاده، مفهوم عضویت درجه‌بندی شده و غیردقیق را مطرح کرد. در این روش درجه عضویت اعضای یک مجموعه مانند مجموعه‌های قطعی محدود به صفر و یک نمی‌شود و می‌تواند همانند آنچه در 
\ref{f_3}
نشان داده شده است شامل درجات عضویت بین صفر تا یک نیز باشد. لطفی‌زاده این مجموعه‌ها را 
\textbf{مجموعه‌های فازی}
 نامید. مفهوم کلمه‌ی فازی به معنای نادقیق و مبهم می‌باشد.
\begin{equation}\label{f_3}
A= \{\  (x, \mu_{A}(x))\ |\ x \in A, \mu_{A}(x) \in [0,1]\  \}
\end{equation} 
مجموعه‌های فازی به صورت زوج مرتب شیء و درجه عضویت نمایش داده می‌شوند. 
 \subsection{عملیات بر روی مجموعه‌های فازی}
 \subsection{‌اعداد فازی}
  \subsubsection{ ‌اعداد فازی قطعه به قطعه درجه دوم}
  \subsubsection{‌اعداد فازی مثلثی}
  \subsubsection{‌اعداد فازی ذوزنقه‌ای}
 \subsection{رابطه‌های فازی}
 \subsection{عملیات بر روی رابطه‌های فازی}
%********************************Fuzzy Logic************************
\section{منطق فازی}
%********************************Other Sctions***********************
\section{کاربرد میانگین فازی برای پیش‌بینی}
\section{تصمیم‌گیری در محیط فازی}
\section{کاربرد کنترل فازی}
\section{مفاهیم اولیه هوش مصنوعی}
\section{جستجوی آگاهانه و ناآگاهانه}
%*******************************References**************************
\bibliographystyle{ieeetr-fa}
\bibliography{Ref}
\end{document}